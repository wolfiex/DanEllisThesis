Humans are social creatures. It is the ability to interact with more than 150 people that allowed for the development of writing and consequenty technology. As technology developed, so did antrhopogenic emissions into the atmosphere. With improvements to the scientific understanding of chemical processes in the atmosphere we are now faced with large amounts of complex data. There are limited ways in which to effectively interprate such data. The simplest lies in using visualisation to proivide a representation in a form intiutive to the user and their expericnes. For the reactions of molecules in the atmosphere, the sociograph structure of a graph was found the simplest to understand.

As chemical schemes which describe the reactions in the atmosphere reach sizes greater than several thousand species manual representation becomes impossible. Instead automatic graph drawing algorithms such of the Force Atlas 2 can be used to create visually appealing graphs. In chemical simulation models, these chemical schemes are represented as a series of numerically stiff differential equations and which are then solved using the Jacobian matrix. This tells us the effect species have on eachother at different timesteps within the model. Such a relational matrix can be used to `weight' the edges of a force graph, changing its shape with respects to the reactions which exist within the model- creating an almost `breathing' like effect as the chemistry moves between daytime and nighttime regimes. 

For schemes which become too large to visually show/comprehend, it is possible to apply centrality metrics to the graph. These are a collection of numerical algorithms that establish important nodes within the network structure. Comparing these with traditional methods of establishing importance, although the centrality metrics agree with the traditional ones, their numerical values are scaled, and therefore they do not provide a higher amount of information.

If instead of discovering importance, we were interested in finding species with high similarity, it is possible to use the infomap graph clustering algorithm.
