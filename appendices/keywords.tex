
\chapter{Chapter Keywords}
This section uses the Term Frequency Inverse Document Frequency to determine the keywords of each chapter - a techneque which has been described in \autoref{ch3} and \cite{frankenstein}. Text size corresponds to the importance of each word.




 \section{ Introduction }
 {\fontfamily{cmtt}\selectfont \parbox{\textwidth}{
\size{34}{ EQN }\size{28}{ EQUATION }\size{19}{ EARTH }\size{19}{ AIR }\size{18}{ ATMOSPHERE }\size{17}{ SPECIES }\size{16}{ OZONE }\size{15}{ POLLUTION }\size{15}{ CLIMATE }\size{14}{ ATMOSPHERIC }\size{14}{ MODEL }\size{13}{ CONCENTRATIONS }\size{13}{ AGO }\size{13}{ REACTION }\size{10}{ CHEMISTRY }\size{10}{ CHEMICAL }\size{10}{ OH }\size{10}{ DT }\size{9}{ CHANGE }\size{9}{ WITHIN }\size{9}{ YEARS }\size{9}{ KM }\size{9}{ GCM }\size{9}{ NNO }\size{8}{ NITROGEN }\size{8}{ OXYGEN }\size{8}{ PRODUCTION }\size{7}{ TROPOSPHERE }\size{7}{ INCREASE }\size{7}{ NUMERICAL }\size{7}{ MECHANISM }\size{7}{ HUMAN }\size{7}{ CYCLE }\size{7}{ ENERGY }\size{7}{ SIGNIFICANT }\size{7}{ PHOTOLYSIS }\size{7}{ IPCC }\size{7}{ HNO }\size{7}{ TIMESCALES }\size{7}{ RESULTED }\size{7}{ PLANET }\size{7}{ POSSIBLE }\size{6}{ LOSS }\size{6}{ LIFE }\size{6}{ SYSTEM }\size{6}{ MODELS }\size{6}{ HOWEVER }\size{6}{ WEATHER }\size{6}{ GAS }\size{5}{ HUMANKIND }\size{5}{ HUMANITY }\size{5}{ DEVCYCLE }\size{5}{ AMBIENT }\size{5}{ REVOLUTION }\size{5}{ POLLUTANTS }\size{5}{ PP }\size{5}{ HOMO }\size{5}{ ESM }\size{5}{ MOTIVATION }\size{5}{ LABORATORY }\size{5}{ HOX }\size{5}{ BILLION }\size{5}{ CHINANOX }\size{5}{ LIFETIME }\size{5}{ MANY }\size{5}{ COMPLEX }\size{5}{ SCIENCE }\size{5}{ UNDERSTANDING }\size{5}{ CH }\size{5}{ MAY }\size{5}{ QUICKLY }\size{5}{ GLOBAL }\size{5}{ DEVELOPMENT }\size{5}{ HV }\size{5}{ THESIS }\size{5}{ EVENTUALLY }\size{5}{ NOX }\size{5}{ RISE }\size{4}{ MASS }\size{4}{ STEADY }\size{4}{ GEOSCHEM }\size{4}{ LEAD }\size{4}{ ONO }\size{4}{ RACE }\size{4}{ TRANSPORTED }\size{4}{ PROBLEMS }\size{4}{ PART }\size{4}{ EXAMPLE }\size{4}{ ALSO }\size{4}{ REACTIONS }\size{4}{ RANGE }\size{4}{ GRAPH }\size{4}{ QUALITY }\size{4}{ REPRESENTATION }\size{4}{ SERIES }\size{4}{ CHANGING }\size{4}{ SIMULATION }\size{4}{ KNOWN }\size{4}{ SOURCE }\size{4}{ SCIENTIFIC }} }

 \section{ The Importance of Visualisation:  From Social Networks to the Atmosphere }
 {\fontfamily{cmtt}\selectfont \parbox{\textwidth}{
\size{30}{ SPECIES }\size{25}{ ARC }\size{23}{ FIGURES }\size{20}{ WITHIN }\size{16}{ ARCS }\size{16}{ REACTIONS }\size{14}{ NUMBER }\size{14}{ CHORD }\size{13}{ MCM }\size{12}{ VISUALISATION }\size{12}{ OH }\size{12}{ RO }\size{11}{ GROUPS }\size{11}{ DATA }\size{10}{ STORYTELLING }\size{10}{ REACTION }\size{10}{ INFORMATION }\size{10}{ NETWORK }\size{9}{ METAPHOR }\size{9}{ MAY }\size{9}{ SEEN }\size{9}{ ITEMS }\size{9}{ HO }\size{9}{ SOCIAL }\size{8}{ USING }\size{8}{ GRAPH }\size{8}{ KNOWLEDGE }\size{8}{ CHEMISTRY }\size{8}{ DIFFERENT }\size{8}{ ALSO }\size{8}{ DIAGRAM }\size{8}{ CERTAIN }\size{8}{ TWO }\size{8}{ PROTOCOL }\size{8}{ NIGHTINGALE }\size{8}{ TRUNK }\size{8}{ PANS }\size{7}{ REPRESENTATION }\size{7}{ RELATIONSHIPS }\size{7}{ BORNEO }\size{7}{ PATHWAYS }\size{7}{ TREE }\size{7}{ HV }\size{7}{ MECHANISM }\size{7}{ ALTHOUGH }\size{7}{ GROUP }\size{7}{ METHODS }\size{7}{ USED }\size{7}{ PLOT }\size{6}{ NARRATIVE }\size{6}{ READERS }\size{6}{ MACE }\size{6}{ HYDROXIDE }\size{6}{ COMPOSITE }\size{6}{ BECK }\size{6}{ POSSIBLE }\size{6}{ REPRESENT }\size{6}{ FUNCTIONAL }\size{6}{ EVENTS }\size{6}{ LONDON }\size{6}{ BRANCHES }\size{6}{ FEATURES }\size{6}{ MATRIX }\size{6}{ MUCH }\size{6}{ MANY }\size{6}{ OFTEN }\size{5}{ THICK }\size{5}{ RADIAL }\size{5}{ IDEAS }\size{5}{ EFFECTIVE }\size{5}{ NEW }\size{5}{ HEAD }\size{5}{ GOSSIP }\size{5}{ CAVE }\size{5}{ TEPHI }\size{5}{ KBPAN }\size{5}{ FLOWCHART }\size{5}{ NATURE }\size{5}{ PRESENTED }\size{5}{ TIME }\size{5}{ SHOW }\size{5}{ EXAMPLE }\size{5}{ PROCESS }\size{5}{ SEVERAL }\size{5}{ ABILITY }\size{5}{ DESIGN }\size{5}{ DUE }\size{5}{ SET }\size{5}{ SIZE }\size{5}{ SUBSET }\size{5}{ CONTAIN }\size{5}{ PATTERNS }\size{5}{ USER }\size{5}{ COMPARING }\size{5}{ COMPARE }\size{5}{ ADJACENCY }\size{5}{ EXPERIENCE }\size{5}{ VISUALISATIONS }\size{5}{ SAPIENS }\size{5}{ METAPHORS }} }

 \section{Applying Visual Analytics to the Atmospheric Chemistry Network}
 {\fontfamily{cmtt}\selectfont \parbox{\textwidth}{
\size{41}{ GRAPH }\size{25}{ NODES }\size{19}{ LAYOUT }\size{18}{ FIGURES }\size{16}{ WITHIN }\size{16}{ NODE }\size{14}{ SPECIES }\size{14}{ GRAPHS }\size{13}{ EDGES }\size{13}{ LAYOUTS }\size{13}{ FORCEDIRECTED }\size{13}{ USING }\size{13}{ EDGE }\size{12}{ REPRESENTATION }\size{11}{ ALGORITHM }\size{11}{ USED }\size{10}{ MERC }\size{10}{ CONFLUENT }\size{10}{ NETWORK }\size{10}{ DISTRIBUTION }\size{10}{ MECHANISM }\size{9}{ DENSITY }\size{9}{ MAY }\size{9}{ MERCATOR }\size{9}{ ANGLE }\size{9}{ CHEMISTRY }\size{9}{ INFORMATION }\size{9}{ REACTIONS }\size{8}{ SEMANTIC }\size{8}{ VISUALISATION }\size{8}{ FORCE }\size{8}{ STRUCTURE }\size{7}{ OPENORD }\size{7}{ TSNET }\size{7}{ POSSIBLE }\size{7}{ CHEMICAL }\size{7}{ DATA }\size{7}{ MCM }\size{7}{ DESIGN }\size{7}{ BEZIER }\size{7}{ CROSSING }\size{7}{ ROUTING }\size{7}{ HU }\size{7}{ DIFFERENT }\size{7}{ APHH }\size{7}{ CH }\size{6}{ LINKS }\size{6}{ CM }\size{6}{ ATLAS }\size{6}{ YIFAN }\size{6}{ SHOWS }\size{6}{ ITEMIZE }\size{6}{ REPRESENT }\size{6}{ ONE }\size{6}{ NUMBER }\size{6}{ DEGREE }\size{6}{ ORTHOGONAL }\size{6}{ FORCEATLAS }\size{6}{ QUADTREE }\size{6}{ DRAWING }\size{6}{ BUTANE }\size{6}{ AREA }\size{6}{ ADDITION }\size{5}{ HIGH }\size{5}{ USER }\size{5}{ EXAMPLE }\size{5}{ PROCESS }\size{5}{ VISUAL }\size{5}{ SYSTEM }\size{5}{ TSNE }\size{5}{ BUNDLING }\size{5}{ ENERGY }\size{5}{ POINTS }\size{5}{ BEIJING }\size{5}{ BEIJINGTEST }\size{5}{ CURVES }\size{5}{ SYNTACTIC }\size{5}{ REPRESENTING }\size{5}{ SHOWN }\size{5}{ ALTHOUGH }\size{5}{ MANY }\size{5}{ SHAPE }\size{5}{ CLUTTER }\size{5}{ DIRECTED }\size{5}{ OFTEN }\size{5}{ METHODS }\size{5}{ CONF }\size{4}{ BEST }\size{4}{ HOWEVER }\size{4}{ NETWORKS }\size{4}{ SINCE }\size{4}{ RESOLUTION }\size{4}{ GENERATED }\size{4}{ SET }\size{4}{ ITEMS }\size{4}{ DESCRIBED }\size{4}{ REDUCE }\size{4}{ ALSO }\size{4}{ LARGE }\size{4}{ FOUND }} }

 \section{ Chemical model diagnostics using graph theory and metrics.  }
 {\fontfamily{cmtt}\selectfont \parbox{\textwidth}{
\size{32}{ SPECIES }\size{21}{ NETWORK }\size{19}{ CENTRALITY }\size{19}{ NODE }\size{19}{ GRAPH }\size{17}{ WITHIN }\size{13}{ JACOBIAN }\size{13}{ PAPERS }\size{13}{ METRICS }\size{12}{ ALGORITHM }\size{12}{ USING }\size{12}{ BETWEENNESS }\size{12}{ MLPREGRESSOR }\size{12}{ METRIC }\size{12}{ CITATION }\size{11}{ CLOSENESS }\size{11}{ USED }\size{11}{ EQN }\size{11}{ CONCENTRATION }\size{11}{ FIGURES }\size{11}{ MATRIX }\size{11}{ MCM }\size{10}{ PAGERANK }\size{10}{ RESULTS }\size{10}{ DATA }\size{10}{ MAY }\size{10}{ EQUATION }\size{9}{ CH }\size{9}{ EQNARRAY }\size{9}{ FREQUENCY }\size{9}{ VALUE }\size{9}{ FLUX }\size{9}{ CHEMISTRY }\size{9}{ SUM }\size{9}{ PAGE }\size{8}{ PT }\size{8}{ MONTH }\size{8}{ GOOGLE }\size{8}{ CAPE }\size{8}{ DOCUMENT }\size{8}{ TIME }\size{8}{ NUMBER }\size{8}{ RANK }\size{8}{ CONCENTRATIONS }\size{7}{ VALUES }\size{7}{ MECHANISM }\size{7}{ MODEL }\size{7}{ NODES }\size{7}{ OBSERVATIONAL }\size{7}{ VERDE }\size{7}{ TEX }\size{7}{ POSSIBLE }\size{7}{ HIGH }\size{7}{ TABLES }\size{6}{ INFLUENCE }\size{6}{ IDF }\size{6}{ AUTHORS }\size{6}{ TFIDF }\size{6}{ IMPORTANT }\size{6}{ OH }\size{6}{ SIMULATION }\size{6}{ PERCEPTRON }\size{6}{ MLP }\size{6}{ SEEN }\size{6}{ ALSO }\size{6}{ DIFFERENT }\size{6}{ MANY }\size{6}{ STRUCTURE }\size{6}{ ANALYSIS }\size{6}{ CHEMICAL }\size{6}{ INITIAL }\size{5}{ LINKS }\size{5}{ OOE }\size{5}{ ADJOINT }\size{5}{ BEND }\size{5}{ TOTAL }\size{5}{ REACTION }\size{5}{ INFORMATION }\size{5}{ CHANGE }\size{5}{ SMALL }\size{5}{ US }\size{5}{ LEFT }\size{5}{ ONE }\size{5}{ METHOD }\size{5}{ EXAMPLE }\size{5}{ BORNEO }\size{5}{ IMPORTANCE }\size{5}{ RANGE }\size{5}{ LINK }\size{5}{ TAB }\size{4}{ SHOWS }\size{4}{ DATASET }\size{4}{ EDGE }\size{4}{ COCITATION }\size{4}{ PREDICTED }\size{4}{ TRADITIONAL }\size{4}{ EVERY }\size{4}{ SHOWN }\size{4}{ MUCH }\size{4}{ SINCE }} }

 \section{ Computational Learning of Species Structure using Visualisation and Vector Clustering}
 {\fontfamily{cmtt}\selectfont \parbox{\textwidth}{
\size{28}{ PCA }\size{22}{ SPECIES }\size{21}{ TSNE }\size{20}{ DATA }\size{17}{ WITHIN }\size{17}{ SMILES }\size{17}{ CLUSTERS }\size{16}{ ALGORITHM }\size{16}{ DR }\size{13}{ VEC }\size{13}{ GROUPS }\size{12}{ ALGORITHMS }\size{12}{ STRUCTURE }\size{11}{ USING }\size{11}{ INPUT }\size{11}{ USED }\size{11}{ DATASET }\size{11}{ DIMENSIONALITY }\size{11}{ FUNCTIONAL }\size{11}{ POINTS }\size{11}{ GRAPH }\size{11}{ SILHOUETTE }\size{10}{ METHODS }\size{10}{ NUMBER }\size{10}{ AUTOENCODER }\size{10}{ REDUCTION }\size{10}{ RANDOM }\size{9}{ CLUSTERING }\size{8}{ LINEAR }\size{8}{ NODE }\size{8}{ DISTRIBUTION }\size{8}{ VECTOR }\size{8}{ QUANTUM }\size{8}{ DIFFERENT }\size{7}{ COEFFICIENT }\size{7}{ CLUSTER }\size{7}{ PRINCIPAL }\size{7}{ DIMENSIONS }\size{7}{ EQUATION }\size{7}{ METHOD }\size{7}{ CH }\size{7}{ FEATURES }\size{7}{ OFTEN }\size{6}{ STRING }\size{6}{ ACTIVATION }\size{6}{ TABLES }\size{6}{ OUTPUT }\size{6}{ ORIGINAL }\size{6}{ INPUTS }\size{6}{ ST }\size{6}{ FINGERPRINTS }\size{6}{ TWO }\size{6}{ MAY }\size{6}{ SINCE }\size{6}{ FEATURE }\size{6}{ MATRIX }\size{6}{ SPACE }\size{6}{ COLOURS }\size{6}{ SET }\size{6}{ TEX }\size{6}{ MCM }\size{6}{ POSSIBLE }\size{6}{ KEYS }\size{5}{ NONLINEAR }\size{5}{ MOLECULAR }\size{5}{ PROBABILITY }\size{5}{ VALUE }\size{5}{ RESULT }\size{5}{ HOWEVER }\size{5}{ COMPONENT }\size{5}{ AE }\size{5}{ DESCRIBED }\size{5}{ REPRESENTING }\size{5}{ BEST }\size{5}{ ONE }\size{5}{ CHEMICAL }\size{5}{ MUCH }\size{5}{ IMPORTANCE }\size{5}{ MQN }\size{5}{ MACCS }\size{5}{ GAUSSIAN }\size{5}{ GREEDY }\size{5}{ CHEMISTRY }\size{5}{ PARAMETERS }\size{5}{ STRUCTURAL }\size{5}{ CONSTRUCTION }\size{4}{ GRADIENT }\size{4}{ TAB }\size{4}{ NETWORK }\size{4}{ GROUP }\size{4}{ ALTHOUGH }\size{4}{ RESULTS }\size{4}{ RANGE }\size{4}{ ACROSS }\size{4}{ COLOUR }\size{4}{ EXAMPLE }\size{4}{ REDUCED }\size{4}{ BOUNDARIES }\size{4}{ THEOREM }\size{4}{ COL }} }