
\section*{A Preface on Humanity and the Climate}
\begin{flushleft}\emph{
The development of humanity is not unlike the chirography of an Aristotelian tragedy. It starts with a simple/primitive species cradling a noble cause - to improve their chances of survival. Here the protagonist (humankind) develops a fatal flaw: an insecurity and latent distruction of their home due to a sudden rise to power.
Having acknowleged this flaw, we now strive to imporve our understanding of the universe, correct past mistakes and stem the tide of inevitable change. \vspace{\baselineskip}\linebreak
With tragedy being an imitation not of humanity, but of action and life, happiness and misery, it is only expected that such a comparison to our current affairs should stir feelings of catharsis when exploring our need for research and scientific advancement.
It is with that I begin this thesis with the begining of the planet, its atmosphere and consequently the beginning of humankind.
}
\end{flushleft}

\section{Whence}
This section describes the intial formation of an atmosphere, how this led to life, and ultimately the human race.
\subsection{Formation of the Atmosphere}
 4.5 billion years ago the Earth began as a disk of dust and gas orbiting our sun. As these gasses move about, resonant drag instability led to the clumping of dust particles, \cite{drag,planet}. As these `clumps' become denser, other forces come in to play, further increasing the size - eventually forming the hot mix of gas and solid which became the Earth.

In its cooling, the newly formed planet began to accumelate primodial gasses from the vollotile componenets of the gas cloud - forming an atmosphere. These gasess were then supplemented through outgassing (volcanic eruptions). At this point in time oxygen was not only absent in the atmopshere, but also had many siks within the Earths anoxidised crust. It was not until oxygenic photosynthesis (\cite{oxygenicphotosynthesis}) that the concentrations of oxygen in the atmosphere started to increase. Eventually the development of multicellular cyanobacteria\footnote{The phylum of phtosynthetic prokaryotic (cells not containing a distinct nucleus) bacteria - e.g. blue-green algae} resulted in biologically induced oxygen accumelating in the atmosphere, \cite{multicellular}. This led to the most significant climate event in the planets history: the Great Oxigenation Event (2.5 billion years ago), \cite{oxidation}. This increase of oxygen allowed oragnisms to become larger and more active, eventually resulting in the human race.

\subsection{Rise of the Homo Spiens (`Wise Man')}
About x million years ago there were many varieties of the homo genus. With the development of the human brain, energy transfer changed. A larger brain required more fuel, and therefore with the development of cooking\footnote{The first known case of indoor air pollution} humans were able to increase their...

This led to the first know source of indoor air pollution.

Ever since we have experienced issues regarding to this... 






With the this increased capability, a language capable of communicating information, allowing for the ability to not only hunt larger prey but also.
Ability to metaphorical, allowed fruther knwoplege transfer , cvave paings and metaphoirical for people over 150 ....

REFERENCES TO OTHER CHAPTERS...
- vis
- accounting via metaphors
- and an interest in science, and atmosphere
