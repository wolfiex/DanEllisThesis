\section{Introduction}

\subsection*{Historical significance}
The established process of trial and error has always underpinned our survival \citep{TrialandError}. Babies are born to rely on a set of sensory reflexes and a framework for physical reasoning \citep{pr}, and with these, we develop methods to navigate the influence of change within a physical, and auditory space \citep{objects}. This method of decision making is reflected in our adult lives with ideas and actions being limited in choice by our intuition and experience \citep{descartes}. In science, we apply a methodological framework consisting of a continuous assessment of scepticism, educated guessing (hypothesizing) and rigorous practical testing. Specialists accrue years of practical and theoretical knowledge within a narrow field and can identify areas of potential gain and futility. Yet even with all prior knowledge, the discovery of new and untested techniques involve the tortuous traipsing through a sea of uncertainty. Such a methods sometimes prove fruitful, through accidental discoveries of items such as x-rays, penicillin, etc. \citep{accidental}; finding novel applications for existing methods such as optical tweezers for chemistry or the abstract field of maths utilised by Einstein [REF], but more often than not end in the constant evolution of a pre-existing project with no clear result. 

\subsection*{Theory and Simulation in Science}

Until recently much of the experimentation possible was limited by resources, levels of knowledge available technology. With the increase of computation power, we have been able to not only increase our understanding but also run theoretical simulations to guide exploratory efforts with an impact on real-world applications \citep{dft,lion,theoreticalbio,drug}. However, as our ability to record and produce data increases, the need for the scientific method diminishes \citep{wired}. Here the application of `big data' tools and algorithms can provide insights and correlations much more compelling than the predictive capabilities of constantly changing models - ``Since all models are wrong the scientist cannot obtain
a "correct" one by excessive elaboration'' - \cite{allmodels}. As our level of attainable technology increases, so does the complexity of the data collected. Modern data-sets tend to be large, complex and highly multivariate. Although this greatly improves the quality of science that may be extracted from them, the difficulty lies in trying to represent it in such a way that we may successfully access the reliability of the results. Since simple bar and line graphs are no longer applicable, one solution falls within a class of unsupervised machine learning techniques called dimensionality reduction (DR).


\subsection*{Chapter Aims}
In \autoref{ch1} we looked at visual representation as a way of understanding complex systems. \autoref{ch2} showed that the chemical properties could be visually inferred from the node-link graph structure of a mechanism. Similarly, \autoref{ch3} and \autoref{ch4} located the presence of important species and clusters of like properties by applying mathematical algorithms to the graph network. As opposed to attempting to visualise complex data, this chapter looks at learning the structure of a chemical species and simplifying it into two dimensions. Here it is possible to extract key features of like-groups through the use of vector clustering, which unlike the graph clustering in \autoref{ch4} works by determining the density between points on a plane.  

The chapter begins with the introduction of the chemical system, and the various methods for representing species structure within it (\autoref{sec:drinput}). Next, we define the dimensionality reduction methods which shall be used to simplify the aforementioned inputs (\autoref{sec:dr}). This is followed by a brief overview of the visualisation methodology (\autoref{sec:visdr}). Finally, all three sections are combined to produce a set of result and conclusions about the use of DR to identify species structure.  

















%