\RequirePackage{fix-cm}
\documentclass[twoside,openleft,reqno,a4paper,final]{book}
%%% Remove the next two lines if you want the figures at their place
%\usepackage[figuresonly,nolists,nomarkers]{endfloat}
%\renewcommand{\processdelayedfloats}{}

\usepackage{soul}%strikeout \st
\usepackage[T1]{fontenc}
\usepackage[utf8]{inputenc}
\usepackage[sectionbib]{natbib}
\usepackage[toc,page]{appendix}
\usepackage[svgnames]{xcolor}
\usepackage{chapterbib}
\usepackage[version=4]{mhchem}
\newcommand{\size}[2]{{\fontsize{#1}{0}\selectfont#2}}
%\usepackage{verbatim}


\usepackage{hyperref,booktabs,algpseudocode,algorithm,booktabs,titlesec,setspace,relsize,import}
\usepackage{amsmath,graphicx,float,subcaption}
\usepackage[bindingoffset=25mm, left=15mm, right=15mm,top=30mm, bottom=20mm]{geometry}
\usepackage[space=true]{grffile}
\usepackage{fancyhdr,booktabs,algpseudocode,algorithm,tikz}
\usepackage{pdflscape}
\usepackage{adjustbox}
\usepackage{afterpage}
\usepackage{chemfig}


%In the preamble H is a hidden column type
\usepackage{array}
\newcolumntype{H}{>{\setbox0=\hbox\bgroup}c<{\egroup}@{}}

\newcommand{\chapterbib}{
\newpage
\bibliographystyle{apalike}
\bibliography{bibtex}
}

%\pagenumbering{roman}  http://www.markschenk.com/tensegrity/latexexplanation.html
 %A4 (210 mm x 297 mm)  https://tex.stackexchange.com/questions/20538/what-is-the-right-order-when-using-frontmatter-tableofcontents-mainmatter
%\addtolength{\textwidth}{12mm}%210
%\addtolength{\evensidemargin}{-4mm}
%\addtolength{\oddsidemargin}{-8mm}
%\addtolength{\textheight}{35mm}%297
\setlength{\parskip}{1.3ex plus 0.2ex minus 0.2ex}
\setlength{\parindent}{0pt}
\setcitestyle{square}
\pagestyle{fancy}
\renewcommand{\chaptermark}[1]{\markboth{\thechapter~-~#1}{}}
 \fancyhf{}
  \fancyhead[LE]{\itshape   \thepage}
 \fancyhead[RE]{\itshape  \nouppercase{\rightmark}} %\nouppercase !
  \fancyhead[LO]{\itshape  \nouppercase{\leftmark }}%thechapter
 \fancyhead[RO]{\itshape  \thepage} %\nouppercase !
\pagestyle{fancy}
\newcommand{\ch}[1]{\MakeUppercase{\ce{#1}}}  % version 1
\newcommand{\multiref}[2]{\autoref{#1}-\ref{#2}} % from - to
\usetikzlibrary{arrows}
\algnewcommand\algorithmicforeach{\textbf{for each}}
\algdef{S}[FOR]{ForEach}[1]{\algorithmicforeach\ #1\ \algorithmicdo}
\def\blankpage{%
      \clearpage%%
      \thispagestyle{empty}%
      \addtocounter{page}{0}%
      \null%
      \clearpage}

%DOUBLE PAGE FIG
\makeatletter
\newcommand*{\twopagepicture}[7]{%

\afterpage{%
\clearpage% flush all other floats
\ifodd\value{page}
%\else% uncomment this else to get odd/even instead of even/odd
    \expandafter\afterpage% put it on the next page if this one is odd
\fi
{%
%\thispagestyle{empty}
    \begin{figure}[#1]
        \raggedleft
        \vspace*{-.035\textheight}
        \adjustbox{trim=0 0cm {.49\width} 0cm,clip}{\includegraphics[height={#6}\textheight,angle=#7]{#3}}%
    \caption{#4}\label{#5}
    \end{figure}
    \thispagestyle{empty}
    \begin{figure}[#1]
        \raggedright
        \vspace*{-.1007\textheight}
        \adjustbox{trim={.49\width} 0cm 0 0cm,clip}{\includegraphics[height={#6}\textheight,angle=#7]{#3}}%
    \end{figure}
    \thispagestyle{empty}

\vfill
    \clearpage
    }
    }
}
\makeatother

% \newcommand{\Autoref}[1]{%
%   \begingroup%
%   \def\chapterautorefname{Chapter}%
%   \def\sectionautorefname{Section}%
%   \def\subsectionautorefname{Subsection}%
%   \def\subsubsectionautorefname{Subsubsection}%
%   \def\paragraphautorefname{paragraph}%
%   \autoref{#1}%
%   \endgroup%
% }

%just do it for all
\def\chapterautorefname{Chapter}%
\def\sectionautorefname{Section}%
\def\subsectionautorefname{Subsection}%
\def\subsubsectionautorefname{Subsubsection}%
\def\paragraphautorefname{Paragraph}%
\def\tableautorefname{Table}
\def\equationautorefname{Equation}


% %titleify section headers
% \usepackage{titlecaps}
% % \usepackage{xparse}
%     %
%
% % Just make titlify python scripts
%
%     \let\svsubsubsection\subsubsection
%     \DeclareDocumentCommand\subsubsection{ s m }{% s = star, m = mandatory arg
%        \IfBooleanTF{#1}{%
%           \svsubsubsection*{\titlecap{#2}}
%        }{%
%          \svsubsubsection{\titlecap{#2}}
%        }%
%     }
%
%     \let\svsubsection\subsection
%     % \def\subsection#1{\svsubsection{\titlecap{#1}}}
%     \DeclareDocumentCommand\subsection{ s m }{% s = star, m = mandatory arg
%        \IfBooleanTF{#1}{%
%           \svsubsection*{\titlecap{#2}}
%        }{%
%          \svsubsection{\titlecap{#2}}
%        }%
%     }

    % \let\svsection\section
    % % \def\section#1{\svsection{\titlecap{#1}}}
    % \DeclareDocumentCommand\section{ s m }{% s = star, m = mandatory arg
    %    \IfBooleanTF{#1}{%
    %       \svsection*{\titlecap{#2}}
    %    }{%
    %      \svsection{\titlecap{#2}}
    %    }%
    % }
    %

    %
    % \let\svparagraph\paragraph
    % \DeclareDocumentCommand\paragraph{ s m }{% s = star, m = mandatory arg
    %    \IfBooleanTF{#1}{%
    %       \svparagraph*{\titlecap{#2}}
    %    }{%
    %      \svparagraph{\titlecap{#2}}
    %    }%
    % }




% location,label,caption,width

\definecolor{c1}{HTML}{004F9D}
\definecolor{c2}{HTML}{00775B}
\definecolor{c3}{HTML}{FF9000}
\definecolor{c4}{HTML}{F71735}
\definecolor{c5}{HTML}{C10053}



\makeatletter
\renewcommand{\maketitle}{

\thispagestyle{empty}
\begin{center}
\pagestyle{empty}
\phantom{.}  %necessary to add space on top before the title
\vspace{1.5cm}

{\Huge \bf \@title\par}
\vspace{2cm}

{\LARGE \@author}\\[.7cm]

{\Large\@date}

\vspace{6cm}

The University of York, Chemistry Department

Submitted in accordance with the requirements\\
    For the degree of Doctor of Philosophy\\ in Atmospheric Chemistry

\vspace{1.3cm}


\textbf{I declare that this thesis is a presentation of original work and I am the sole author. This work has not previously been presented for an award at this, or any other, University. All sources are acknowledged as References.}

\vspace{1cm}
\includegraphics[width=.3\textwidth]{UOY.eps}

\end{center}
}\makeatother






\title{Understanding Atmospheric Chemistry Using Graph-Theory, Visualisation and Machine Learning}
\author{Daniel Ellis}

\date{March 2020}


\begin{document}
% \newgeometry{oneside}

\frontmatter
\titleformat{\paragraph}[hang]{\normalfont\normalsize\bfseries}{\theparagraph}{1em}{}
\titlespacing*{\paragraph}{0pt}{3.25ex plus 1ex minus .2ex}{1em}



\setcounter{secnumdepth}{3}
\setcounter{tocdepth}{3}

%todepth5 is paragraph
%\cleardoublepage
%\setcounter{page}{0}
%\setcounter{chapter}{0}
 \maketitle




\cleardoublepage{}
\setlength{\footnotesep}{0.5cm}
\raggedbottom %group writing to top of page!

%\blankpage
\restoregeometry
\vspace*{0.15\paperheight}
\begin{center}
\begin{quotation}
    \centering
  \large{\emph{\textbf{Veritatem inquirenti, semel in vita de omnibus,\\ quantum fieri potest, esse dubitandu:}\\
 In order to seek truth, it is necessary once in the course of our life, to doubt, as far as possible, of all things. }  }
\vspace{\baselineskip}\linebreak
  \begin{flushright}\small{
  - Descartes, Rene, \textit{Principles of Philosophy}
  }
\end{flushright}
 \end{quotation}
\end{center}


\newpage


\chapter*{Abstract}
\parbox{.75\textwidth}{
Atmospheric chemistry mechanisms play a pivotal role in our understanding of societal problems such as air pollution, climate change and stratospheric ozone loss. This thesis explores the benefits of representing these mechanisms in terms of a mathematic graph (or network) which connects species (nodes) through reactions (edges). Using the Master Chemical Mechanism run using the Dynamically Simple Box Model of Atmospheric Chemical Complexity we run simulations under a number of different representative scenarios and use graph theory and machine learning to visualise, understand and analyse the underlying chemical processes in the atmosphere.

Chapter one discusses the use of various methods in the presentation of complex datasets. Chapter two applies the sociograph framework to atmospheric mechanisms and determines the best way in which to present these. Chapter three takes a more mathematical approach, comparing the results of graph centrality metrics applied to model simulation resuts against more traditional diagnostic methods. The use of graph theory is continued in Chapter four, where graph clustering and natural language processing is used to identify pairs of nodes with similar patterns. Finally Chapter five ventures into the field of chemical informatics, and looks at the use of different representations of species structure within machine learning models (PCA, t-SNE and AutoEncoders) with an aim to merging the content of this thesis into a Graph Convoluted Neural Network in future work.

Ultimately it is found that using a graph-theory approach can prove highly beneficial in the understanding and explanation of chemical mechanisms, but should not (as of yet) be used in substitution of existing investigation and reduction (sensitivity)  analysis methods.

}





 % \begin{center}

    \chapter*{Acknowlegements}


\parbox{.75\textwidth}{
 First and foremost, I would like to thank my family, for which I wish I had spent more time with rather than being squirrelled away in an office. I could have gotten here only with your support for which I am grateful.\\

 Next, are my supervisors Andrew and Mat - mainly putting up with me but also being highly supportive throughout the PhD.\\

 A special mention also needs to go out to all my friends and colleagues at '(\href{www.westminsterboatingbase.co.uk} {Westminster Boating Base}) who helped keep me sane irrespective to the severe watersport withdrawal, kept me employed and invited me to the annual France whitewater trip each year.\\

 On a similar note, thanks to all my university friends and collegues  (Rosie, Tomas, Ben, Mike, Killian, Pete, Peter etc.) with whom I could bounce ideas and discuss random work adjacent projects when needed.\\

 Additionally the Earth0 HPC cluster, for not failing until the very end, even after it was no longer officially in use.\\

 Finally, I am not sure whether to thank or apologise to those unfortunate enough to have to read any part of this work, but I am grateful anyway.\\

}
% \end{center}
% \newpage



\tableofcontents
\listoffigures
\listoftables
\newpage
\mainmatter



%
%
% % Introduction 0
% 

\chapter{ Introduction }\label{ch0}


%\cleardoublepage{}
\blankpage
\restoregeometry
\vspace*{0.15\paperheight}


\begin{center}
\begin{quotation}
  \large{\emph{\textbf{``In the beginning the Universe was created.
This has made a lot of people very angry and been widely regarded as a bad move''} }  }  \\
  \begin{flushright}
  - Douglas Adams, \textit{The Restaurant at the End of the Universe}
  \end{flushright}
 \end{quotation}
\end{center}
\doublespacing
% \onehalfspacing
\newpage

%
%\subimport{intro/}{preface.tex}

%\subimport{intro/}{health.tex}

%\subimport{intro/}{chemistry.tex}

%\subimport{intro/}{model.tex}

%\subimport{intro/}{thesislayout.tex}






\subimport{intro/}{combigned.tex}

\chapterbib

%
% % Ch1 0
%  


\chapter{ The Importance of Visualisation:  From Social Networks to the Atmosphere }\label{ch1}

%\cleardoublepage{}
\blankpage
\restoregeometry
\vspace*{0.15\paperheight}



\begin{center}
\begin{quotation}
  \large{\emph{\textbf{``If you really want to understand something, the best way is to try and explain it to someone else. That forces you to sort it out in your mind. ... By the time you’ve sorted out a complicated idea into little steps that even a stupid machine can deal with, you’ve learned something about it yourself.\footnote{ Omitted at ellipsis : ``And the more slow and dim-witted your pupil, the more you have to break things down into more and more simple ideas. And that’s really the essence of programming.''}''} }  }  \\
  \begin{flushright}
  - Douglas Adams, \textit{Dirk Gently's Holistic Detective Agency}
  \end{flushright}
 \end{quotation}
\end{center}
\doublespacing
% \onehalfspacing
\newpage

%

\subimport{visintro/}{intro.tex}

\subimport{visintro/}{network.tex}


\chapterbib

% % %
% % Chapter 2
% 

\chapter{Applying Visual Analytics to the Atmospheric Chemistry Network}\label{ch2}


%\cleardoublepage{}
\blankpage
\restoregeometry
\vspace*{0.15\paperheight}



\begin{center}
\begin{quotation}
  \large{\emph{\textbf{`` I have a notion that when the mind is thinking, it is simply talking to itself, asking questions and answering them. ''} }  }  \\
  \begin{flushright}
  - Socrates, \textit{The collected dialogues of Plato} 
  \end{flushright}
 \end{quotation}
\end{center}
\doublespacing
% \onehalfspacing
\newpage

% 
 
%\subimport{visanalytics/}{graph.tex}
%\subimport{visanalytics/}{layouts.tex}
%\subimport{visanalytics/}{semantic.tex}
%\subimport{visanalytics/}{result.tex}
 
\subimport{visanalytics/}{combigned.tex} 

\chapterbib







% %
% % Chapter 3
% 

\chapter{ Chemical model diagnostics using graph theory and metrics.  }\label{ch3}


%\cleardoublepage{}
\blankpage
\restoregeometry
\vspace*{0.15\paperheight}


\begin{center}
\begin{quotation}
  \large{\emph{\textbf{``The complexities of cause and effect defy analysis.''} }  }  \\
  \begin{flushright}
  - Douglas Adams, \textit{Dirk Gently's Holistic Detective Agency}
  \end{flushright}
 \end{quotation}
\end{center}
\doublespacing
% \onehalfspacing
\newpage

% 
%\subimport{model_diagnostics/}{intro.tex}
%\subimport{model_diagnostics/}{centrality.tex}
%\subimport{model_diagnostics/}{chemistry.tex}
%\subimport{model_diagnostics/}{graphconstruction.tex}
%\subimport{model_diagnostics/}{casestudy.tex}
%\subimport{model_diagnostics/}{conclusion.tex}
 
\subimport{model_diagnostics/}{combigned.tex} 

\chapterbib







%
% %
% % Chapter 4 - done
%  

\chapter{ Chemical mechanism stratification and analysis using ML and graph clustering. }\label{ch4}

%\cleardoublepage{}
\blankpage
\restoregeometry
\vspace*{0.15\paperheight}


\begin{center}
\begin{quotation}
  \large{\emph{\textbf{``Entities should not be multiplied beyond necessity.''} }  }  \\
  \begin{flushright}
  - William of Ockham, \textit{Summa Logicae}
  \end{flushright}
 \end{quotation}
\end{center}
\newpage

\doublespacing
% \onehalfspacing

 \subimport{mechanism_lumping/}{intro.tex}
 \subimport{mechanism_lumping/}{reduction.tex}
  \subimport{mechanism_lumping/}{datasetup.tex}

  \subimport{mechanism_lumping/}{infomap.tex}
 \subimport{mechanism_lumping/}{nlp.tex} % 
 % 
  \subimport{mechanism_lumping/}{res.tex}
 
\chapterbib


%
% Chapter 5
 

\chapter{ Computational Learning of Species Structure using Visualisation and Vector Clustering}\label{ch5}
% \vspace{-1cm} {\Large Learning species structure using unsupervised machine learning. }


%\cleardoublepage{}
\blankpage
\restoregeometry
\vspace*{0.15\paperheight}


\begin{center}
\begin{quotation}
  \large{\emph{\textbf{``So, in the interests of survival, they trained themselves to be agreeing machines instead of thinking machines. All their minds had to do was to discover what other people were thinking, and then they thought that, too.''} }  }  \\
  \begin{flushright}
  - Kurt Vonnegut, \textit{Breakfast of Champions}
  \end{flushright}
 \end{quotation}
\end{center}
\doublespacing
% \onehalfspacing
\newpage

%
%\subimport{dr/}{newintro.tex}

%\subimport{dr/}{begin.tex}
%\subimport{dr/}{drmethods.tex}
%\subimport{dr/}{visclusters.tex}
%\subimport{dr/}{results1.tex}


%%\subimport{dr/}{appendix.tex}


\subimport{dr/}{combigned.tex}

\chapterbib

%
%
% % Conclusion
%  

\chapter{ Conclusion }\label{ch6}
% \vspace{-1cm} {\Large Learning species structure using unsupervised machine learning. }


%\cleardoublepage{}
% \blankpage
% \restoregeometry
% \vspace*{0.15\paperheight}

%
% \begin{center}
% \begin{quotation}
%   \large{\emph{\textbf{``So, in the interests of survival, they trained themselves to be agreeing machines instead of thinking machines. All their minds had to do was to discover what other people were thinking, and then they thought that, too.''} }  }  \\
%   \begin{flushright}
%   - Kurt Vonnegut, \textit{Breakfast of Champions}
%   \end{flushright}
%  \end{quotation}
% \end{center}
% \doublespacing



% \onehalfspacing

%
% \newpage

%
\subimport{conclusion/}{combigned.tex}
%
% \chapterbib

% %

\cleardoublepage\makeatletter\@openrightfalse\makeatother
\begin{appendices}
% %
\subimport{dr/}{appendix.tex}
\subimport{appendices/}{misc.tex}
\subimport{appendices/}{keywords.tex}

% \chapterbib
\end{appendices}




\newpage
%
%\bibliographystyle{apalike}
%\bibliography{bibtex}
%% \bibliographystyle{unsrt}



\end{document}
