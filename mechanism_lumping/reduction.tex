
\section{Mechanism Reduction}







%The methods of Jacobian analysis and
%overall rate sensitivity analysis proved to be efficient and capable of removing the majority of redundant reactions and
%species in the scheme across a wide range of conditions relevant to the polluted troposphere.\\
%
%The use of the
%quasi-steady state approximation (QSSA) proved to be an extremely successful method of removing the fast time-scales
%within the system, as demonstrated by a local perturbation
%analysis at each stage of reduction. QSSA species were automatically selected via the calculation of instantaneous QSSA
%
%whitehouse pt 1 
%

As discussed, the first step to simplifying a complex task involves the partitioning data into categories. For a mechanism we begin by looking at the reaction or species which are related to the area that is being researched. Items are partitioned into important, needed and redundant categories (described below). 

\begin{itemize}
    \item \textbf{Important} - reactions or species directly related to the topic / outcome we are interested in
    \item \textbf{Needed} - reactions/species required by the important species, such that they may perform their desired function
    \item \textbf{Redundant} - those we may remove with little or no consequence to the final outcome of the model. 
\end{itemize}



\subsection{Reaction Removal}
Since atmospheric chemical mechanism form a numerically stiff system, a reduction in the number of reactions within a mechanism leads to a reduction in the computational burden experienced by a model each iteration forwards in time. Classically the identification of important reactions may be accomplished through the use or rate of production and loss analysis (SEC REF). This allows us to filter reactions contributing less than 5\% to the formation of any species we are interested in. Other methods using principle component analysis of the sensetivity of species (PCAS) also exists and are discussed in \cite{PCAS}.


\subsection{Species removal}
Similar to reaction removal, species removal is useful not only because it reduced the size of the jacobian, but the removing or combining of species inherently reduces or simplifies the reactions within a mechanism.  This method also has added benefit of reducing the size of the jacobian matrix used to propagate the chemical system forwards. For large systems which do not use a sparse framework, storing an $n**2$ matrix in memory can prove difficult.

There are many methods of species reduction that are possible. The simplest of these is through the use of trial and error\footnote{A tried an tested method for scientific discovery.} \citep{tur1990} (Method 1). Here the consuming reactions for a species are removed, and if the resulting deviation in results between the full and reduced mechanism is small, ther results are kept. The main downside to this is that it only works on a per-species level, which may be very resource consuming for large mechanisms.

Alternatively it is also possible to remove species whose reactions are much slower than the rate determining steps of a mechanism, \citep{frenk}. Although this method is useful in the field of combustion modelling, this has most likely been done for atmospheric chemical mechanisms. 

Finally the use of jacobian sensetivity analysis has intensively been used in the determination of which species may be removed from a mechanism.    
\cite{whitehouse1} states this to be a `capable' and `efficient' method for removing most redundant reactions and species from the MCM. Use of a log-normalised Jacobian to determine which species can be removed is found in the connectivity method \autoref{connectivity,cm}. 
Here the influence in changing a 1\%  change in a species concentration has on the concentration of important species can be determined by


\begin{equation}
B_i  = \sum_j(({y_i}/{f_i})({\partial f_i}/{\partial y_i}))^2 \label{connectivity}
\end{equation}

 where $({y_i}/{f_i})({\partial f_i}/{\partial y_i})$ is element $i$ of the normalised Jacobian. Through an iterative process species with a low contribution to our important species can be dound and removed.




\subsection{Lumping}



Rather than removing species or reactions from a mechanism we may combine them to form a new composite species. This is species lumping. To do this we must first consider how we determine species that are to be joined together, and then how their grouped reactions will contribute to every other species it reacts with. Some of the more general types of lumping styles are outlined below. 


\subsubsection{Chemical Lumping}
Mechanisms follow protocols in their generation. This produces reaction styles that many like-structured species follow in their degregation. In determining such classes we may be able to generalise like-species reactions and group them together as one. Taking the CRI mechanism as an example, this has taken the Ozone production capability as a feature of interest. The rato of CC and CH bonds are used to determine a species oxidation possibility ... An example of this are species such as CARB9  ... some examples and what the names mean \\

As this type of lumping has already been performed on our starting mechanism, we shall not be applying it any further. 


\subsubsection{Linear}
\subsubsection{Lifetime}
\subsubsection{Quasi Steady State Approximation (QSSA)}
QSSA works on the axiom that the the flux through a species is 0  - Use louise whitehouses thesis here (better description that the analysis to kinetic reactions book)
