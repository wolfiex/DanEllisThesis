
\section{Introduction}

In the previous chapters we have discussed visualisation and its role in bridging the gap between data and understanding. We have applied centrality metrics to a chemical network to tell us what species are of importance and experimented in getting various machine learning models to learn the chemical structure of the species involved in a mechanism. In this final research chapter we provide a (very) brief overview on mechanism reduction and propose two methods for simplifying the chemistry within a network. 

Atmospheric Chemical Mechanism sizes have been increasing steadily in size over the last 10 years \autoref{CH@FIG!!!!!!!!}. With the ability to automate mechanism construction, mechanisms with billions of species can be generated using a nubmer of pre-defined of rules - the mechanism protocol. Unfortunately as with large data, large mechanisms can pose a problem for the computation, visualisation and analysis of the chemistry. Having looked a methods to represent and query a mechanism, we now explore the different ways in which it may be simplified. 


