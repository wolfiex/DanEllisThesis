
\section{Introduction}

In the previous chapters we have discussed visualisation and its role in bridging the gap between data and understanding. We have applied centrality metrics to a chemical network to tell us what species are of importance and experimented in getting various machine learning models to learn the chemical structure of the species involved in a mechanism. In this final research chapter we provide a brief overview on mechanism reduction and propose two methods for simplifying the chemistry within a network. 

Science often deals with the problem of understanding complexity. Often this may be accomplished through organisation and partitioning, for example the learning of a new skill, or the parallelisation of a large mathematical problme. In cases where such methods fail, we are forced to `disregard' complexity. It is common to aproximate an atom as a sphere or the value $\pi$ as 3 with little consequence to the overall result of a calculation. The process of lumping has long been used replace a complex, changing process (e.g Quantum Mechanics or Boundary Layer Fluid Dynamics) with a simpler constant process, \citep{approx}. In such cases an approximate analysis may be far more useful than a lengthy exact solution, or none at all. 

Similar problems of complexity can also be seen within the chemistry of the atmosphere. An example is seen within the Master Chemical Mechanism\footnote{Version 3.3.1 .} (MCM), \citep{mcm}, containing 1228 \ch{RO2} reactions. If written explicitely all \ce{RO2-RO2} interactions would result in the addition of 1507984 reactions. Instead the MCM overcomes this problem by createing an \ch{ro2} pool, with which all \ch{RO2} species react. This results in a mechanism which produces similar results, but only contains 0.000814 of the total possible \ch{ro2-ro2} reactions.

However even with such simplifications atmospheric chemical mechanisms have been increasing in size over the last 10 years, \autoref{fig:mcmgoogleREF}REF. With the ability to automate their construction, mechanisms with species numbers of the millions become possible. Although the existence of more-explicit mechanisms may improve the quality of science produced, they can cause problems for efficient compuation, diagnosis and anlysis. This chapter shall look at two methods in which we may simplify a mechanism by grouping similar species together.


