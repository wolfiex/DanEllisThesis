

\section{Reduction through Lifetime}





\subsubsection{Calculating the lifetime}
Within models a species lifetime is regarded as the time taken for its concentration to halve [ref]. This works on the assumption that the species is not produced, and that rate coefficients and other constants remain constant. For a first order decay of sample \autoref{eqn}, we can represent the decay using \autoref{decay}, showing that the half life is independent of initial concentration. 

\begin{equation}
A \rightarrow{^k} B
\label{eqn}
\end{equation}

\begin{equation}
s(t) = a_0 \exp(-kt) \\
\frac{a(t)}{a_0} = \exp(-kt) \\
$$linearised this gives$$
\ln (\frac{a(t)}{a_0}) = -kt
$$ after $\tau_{1/2}$ the concentration is equal to $a_0/2$ of initial rate $a_0$, which gives $$
\ln(\frac{\frac{a_0}{2}}{a_0}) = \ln(\frac{1}{2}) = \ln(2^{-1}) = -\ln2 = k\tau_{1/2} 
$$$$
\tau_{\frac{1}{2}} = \frac{\ln 2}{k}
\label{decay}
\end{equation}

In species of the first order only, this may simplified to 
\begin{equation}
a(t) = a_0 \exp (t  \sum_j k_j )
$$ and therefore the half life may be written as the reciprocal sum of rate coefficients: $$
\tau_A = 1 / \sum_j k_j
\end{equation}

and is how lifetime is calculated for photochemical species [ref! pillin and seakins]. An alternative method for half life calculation may be obtained using the diagonal (self reference) of a Jacobian matrix ,\cite{kinetics}:

\begin{equation}
\tau_1 = - \frac{1}{J_{ii}}
\end{equation} 

This value will usually be negative unless a species does not contain a consuming reaction, then it will be zero. 


The xxxxx method of reduction consists of the isolation of species with similar lifetimes and reactions as a means of lumping. In doing so the ... etc 


\subsection{Comparing Magnitude and Direction}
Since the photolysis reactions in a model change the resultant rates, and thus flux of a species depending on the azimuthal angle related to the time of day, we not only want to compare species with the same magnitude, they also need to match the profile as they change. To do this we may represent all pariwise species matches on a latent space representing the size and angle between their temporal vectors. This is done through using the euclidean distance on the $x$ axis, and cosine distance $y$ on the $y$. 

\subsubsection{Euclidian distance}
This is the simplest method of vector comparison and works by calculating the distance between all points in two vectors. For the vectors

\begin{equation}
v1 = [ a,b,c, \dots n ] 
$$$$
v2 = [ i,j,k, \dots z ]
\end{equation}

This can be done using pythoagoras' theorem in \autoref{euclid}:

\begin{equation}
e_{dist}  = \sqrt{(a-i)^2 + (b-j)^2 + (c-k)^2 + \dots + (n-z)^2}
\label{euclid}
\end{equation}

This transformation converts the straight line distance between each vector into metric space, allowing us to represent the difference in their magnitudes as a single scalar. Unfortunately as this requires the difference between all permutations of rows, it cannot be done as a single operation, but as multiple. \\

APPLICA"tiion

\subsubsection{Cosine Distance} 

Similarly if we wish to calculate the angle between two vectors we may use the cosine difference. In starting with the definition of the dot product 

\begin{equation}
v1 \cdot v2 = \|v1\|\|v2\| \cos \theta
$$this may be arranged$$
\cos \theta = \frac{ v1 \cdot v2}{\|v1\|\|v2\|}
\end{equation}

Since this does not work for the triangle ? inequality, we need to normalise each vector before calculating the cosine distance. The merits of this come from  ... which makes its application comparing the similarity between texts or documents of different sizes very popular (REF!). \\


COMPARE force graph of cosine differences and force graph of euclidean distances. Colour ones close to eachother. \\


